\documentclass[a4paper]{article}

\usepackage[english]{babel}
\usepackage[utf8x]{inputenc}
\usepackage{amsmath}
\usepackage{graphicx}
\usepackage[colorinlistoftodos]{todonotes}
\usepackage{mathtools}
\usepackage{amssymb}

\title{Symmetric Positive-Definite matrix bounds}
\author{Emil Kerimov}
\date{\today}
\begin{document}
\maketitle

\newtheorem{theorem}{Theorem}[section]
\newtheorem{corollary}{Corollary}[theorem]
\newtheorem{lemma}[theorem]{Lemma}
\newtheorem{definition}{Definition}[section]

\section{Motivating Example}
Suppose there are three assets, $a,b,c$ , by knowing the volatility of each of the assets, and the correlation between assets $a$ and $b$, as well as the correlation between assets $b$ and $c$, what is the interval of valid correlations between and assets $a$ and $c$?

This document presents one general method of addressing this problem, and solves the case of three assets analytically.

\section{Definition}

\begin{definition}\label{PD}
A positive-definite matrix is defined as TODO.
\end{definition}

\begin{definition}\label{PSD}
A semi-positive-definite matrix is defined as TODO.
\end{definition}

\begin{definition}\label{principle minor}
A principle minor is defined as TODO.
\end{definition}

\section{Background Theory}
We make use of some known properties of positive-definite (PD) and positive-semi-definite (PSD) matrices. 

\begin{theorem} \label{PD implies positive eigenvalues} 
A PD matrix must have all positive eigenvalues.

Proof:
\begin{gather*}
\shortintertext{From the definition of PD}
TODO
\end{gather*}
\end{theorem}

\begin{theorem} \label{PD implies positive principle minors} 
A PD matrix must have all positive principle minors.

Proof:
\begin{gather*}
\shortintertext{From the definition of PD}
TODO
\end{gather*}
\end{theorem}

\section{Main Idea}
All principle majors of a PD matrix, must be positive. Therefore compute them all, and figure out bounds on the unknown quantities. 

\subsection{Two Asset}
The case of $n=2$ is trivial, due to the definition of correlation being bounded by $-1$ and $1$. 

\begin{gather*}
|\rho_{a,b}| \leq 1 \\
|\frac{\sigma_{a,b}}{\sigma_a \cdot \sigma_b}| \leq 1 \\
\shortintertext{Since $\sigma_a$ and $\sigma_b$ are positive by definition} \\
|\sigma_{a,b}| \leq \sigma_a \cdot \sigma_b
\end{gather*}

We show that the positive principle minors method arrives at the same bound. 

Let matrix $C$ be a $2$x$2$ matrix, 
\begin{equation}
C = \begin{pmatrix}
\sigma_a^2 & \sigma_{ab} \\
\sigma_{ab} & \sigma_b^2
\end{pmatrix}
\end{equation}

\subsection{3x3}
We can  

\begin{equation}
C = \begin{pmatrix}
1 & \rho_{ab} & \mathbf{\rho_{ac}} \\
\rho_{ab} & 1 & \rho_{bc} \\
\mathbf{\rho_{ac}} & \rho_{bc} & 1
\end{pmatrix}
\end{equation}

\begin{theorem} \label{three} 
\begin{equation}
\boxed{
|\mathbf{\rho_{ac}} - \rho_{ab} \cdot \rho_{bc}| \leq \sqrt{(1 - \rho_{ab}^2) \cdot (1 - \rho_{bc}^2)}
}
\end{equation}

Proof:
\begin{gather*}
\shortintertext{Det(C) >= 0}
TODO
\end{gather*}
\end{theorem}

\subsection{4x4}
We can  

\begin{equation}
C = \begin{pmatrix}
1 & \rho_{ab} & \rho_{ac} & x \\
\rho_{ab} & 1 & \rho_{bc} & \rho_{bd} \\
\rho_{ac} & \rho_{bc} & 1 & \rho_{cd} \\
x & \rho_{bd} & \rho_{cd} & 1
\end{pmatrix}
\end{equation}

\begin{theorem} \label{four} 
Bounds on $x=\rho_{ad}$
\\
\boxed{
\begin{gathered}
% \boxed{
|\mathbf{\rho_{ad}} - \frac{(\rho_{a,b} \cdot \rho_{b,d} + \rho_{a,c} \cdot \rho_{c,d}) - \rho_{b,c} \cdot (\rho_{a,b} \cdot \rho_{c,d} + \rho_{a,c} \cdot \rho_{b,d}) }{1 - \rho_{b,c}^2} | \leq (1 - \rho_{b,c}^2) \sqrt{D}
\\
D = (1 - \rho_{a,b}^2 - \rho_{a,c}^2 -\rho_{b,c}^2 + 2 \cdot \rho_{a,b} \cdot \rho_{a,c} \cdot \rho_{b,c}) \cdot (1 - \rho_{b,c}^2 - \rho_{b,d}^2- \rho_{c,d}^2+ 2 \cdot \rho_{b,c} \cdot \rho_{b,d} \cdot \rho_{c,d})
\end{gathered}
}
   
Proof:
\begin{gather*}
Det(C) \geq 0 
\\
1 - (\rho_{c,d}^2 + \rho_{b,c}^2 + \rho_{b,d}^2 + \rho_{a,b}^2 + \rho_{a,c}^2) 
\\
+ 2 \cdot (\rho_{b,c} \rho_{b,d} \rho_{c,d} 
+ \rho_{a,b} \rho_{b,c} \rho_{a,c}) 
\\
- 2 ( \rho_{a,b} \rho_{b,d} \rho_{a,c} \rho_{c,d} )
\\
+ (\rho_{a,b}^2 \rho_{c,d}^2 + \rho_{a,c}^2 \rho_{b,d}^2)
\\
+ 2x (\rho_{a,b} \rho_{b,d} + \rho_{a,c} \rho_{c,d})
\\
- 2x ( \rho_{a,b} \rho_{b,c} \rho_{c,d} 
+ \rho_{a,c} \rho_{b,d} \rho_{b,c} )
\\
+ x^2 ( \rho_{b,c}^2 -1) \geq 0
\\
\shortintertext{This is just a quadradic equation}
\\
a = \rho_{b,c}^2 -1 
\\
b = 2( \rho_{a,b} \rho_{b,d} + \rho_{a,c} \rho_{c,d}) - 2 ( \rho_{a,b} \rho_{b,c} \rho_{c,d} 
+ \rho_{a,c} \rho_{b,d} \rho_{b,c} ) 
\\
c = 1 - (\rho_{c,d}^2 + \rho_{b,c}^2 + \rho_{b,d}^2 + \rho_{a,b}^2 + \rho_{a,c}^2) 
\\
+ 2 \cdot (\rho_{b,c} \rho_{b,d} \rho_{c,d} 
+ \rho_{a,b} \rho_{b,c} \rho_{a,c}) 
\\
- 2 ( \rho_{a,b} \rho_{b,d} \rho_{a,c} \rho_{c,d} )
\\
+ (\rho_{a,b}^2 \rho_{c,d}^2 + \rho_{a,c}^2 \rho_{b,d}^2)
\\
ax^2 + bx + c \geq 0
\\
|2ax + b| \leq \sqrt{b^2 - 4ac}
\\
\shortintertext{Dividing all sides by 2}
a = \rho_{b,c}^2 -1 
\\
b = ( \rho_{a,b} \rho_{b,d} + \rho_{a,c} \rho_{c,d}) - ( \rho_{a,b} \rho_{b,c} \rho_{c,d} 
+ \rho_{a,c} \rho_{b,d} \rho_{b,c} ) 
\\
c = 1 - (\rho_{c,d}^2 + \rho_{b,c}^2 + \rho_{b,d}^2 + \rho_{a,b}^2 + \rho_{a,c}^2) 
\\
+ 2 \cdot (\rho_{b,c} \rho_{b,d} \rho_{c,d} 
+ \rho_{a,b} \rho_{b,c} \rho_{a,c}) 
\\
- 2 ( \rho_{a,b} \rho_{b,d} \rho_{a,c} \rho_{c,d} )
\\
+ (\rho_{a,b}^2 \rho_{c,d}^2 + \rho_{a,c}^2 \rho_{b,d}^2)
\\
|ax + b| \leq \sqrt{b^2 - ac}
\\
\shortintertext{Simplifying the square root}
b^2 - ac = (a,b)^2 (b,d)^2 + (a,c)^2 (c,d)^2 + (b,c)^2 (c,d)^2 +(b,c)^2 (b,d)^2
\\
 + (b,c)^2 (a,b)^2 + (b,c)^2 (a,c)^2 - (a,b)^2 (c,d)^2 -(a,c)^2 (b,d)^2 
\\
+(2*(b,c)^2 + (c,d)^2 + (b,d)^2 + (a,b)^2 + (a,c)^2) -1 
\\
+
4 (a,b)(b,c)^2 (c,d)(a,c)(b,c) - 2((b,c)(b,d)(c,d) + (a,b)(b,c)(a,c)) - (b,c)^4 
\\
- 2((a,b)^2 (b,d)(b,c)(c,d) + (a,b)(b,d)^2(b,c) (a,c) + (a,c)(c,d)^2 (a,b) (b,c) 
\\
+ (a,c)^2 (c,d) (b,c) (b,d)) - 2 ( (b,c)^3 (b,d) (c,d) + (b,c)^3 (a,b) (a,c))
\end{gather*}
\end{theorem}

\end{document}