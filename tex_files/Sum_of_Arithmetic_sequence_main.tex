\documentclass[a4paper]{article}

\usepackage[english]{babel}
\usepackage[utf8x]{inputenc}
\usepackage{amsmath}
\usepackage{graphicx}
\usepackage[colorinlistoftodos]{todonotes}
\usepackage{mathtools}

\title{Sequences}
\author{Emil Kerimov}
\date{\today}
\begin{document}
\maketitle

\newtheorem{theorem}{Theorem}[section]
\newtheorem{definition}{Definition}[section]

\section{Arithmetic Sequence}\label{sec:arithmetic-sequence}

\begin{definition}\label{Arithmetic def}
Given a starting value of $a_0$ and a constant difference value of $d$ the arithmetic sequence is defined as follows.
\[\boxed{
a_n = a_0 + n \cdot d
}
\]
\end{definition}

\subsection{Sum of Arithmetic sequence}\label{subsec:sum-of-arithmetic-sequence}


\begin{theorem}
\begin{equation} \label{eq:sum-of-arithmetic-sequence}
\sum\limits_{k=0}^{n} a_k = \frac{(n+1)}{2} ( 2 a_0 +n \cdot d )
\end{equation}

Proof
\begin{gather*}
\intertext{Define $S_n$ such that:}
S_{n} = \sum_{k=0}^{n} a_k 
\\
\intertext{Note that}
2 S_{n} = \sum_{k=0}^{n} a_k + \sum_{k=0}^{n} a_{n-k} = \sum_{k=0}^{n} \Big( a_k + a_{n-k} \Big)
\\
\intertext{Using the definition we see $a_k + a_{n-k} = \big( a_0 + k \cdot d \big) + \big( a_0 + (n-k) \cdot d \big) =  2 a_0 +n \cdot d $ which does not depend on $k$}
\\
2 S_{n} = \sum_{k=0}^{n} \Big( 2 a_0 +n \cdot d \Big) = (n+1) \Big( 2 a_0 +n \cdot d \Big)
\\
S_{n} = \frac{(n+1)}{2} ( 2 a_0 +n \cdot d )
\end{gather*}
\end{theorem}

\subsection{Example: Sum of Natural Numbers}\label{Sum of Natural Numbers}
Prove that $$\sum\limits_{i=0}^{n} i = 0+ 1+2+3+...+n = \frac{n(n+1)}{2}$$

Proof:
Set $a_0 = 0$ and $d=1$
$$\sum\limits_{i=0}^{n} i
=
\frac{(n+1)}{2} (n )$$

\end{document}