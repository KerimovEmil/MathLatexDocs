\documentclass[a4paper]{article}

\usepackage[english]{babel}
\usepackage[utf8x]{inputenc}
\usepackage{amsmath}
\usepackage{graphicx}
\usepackage[colorinlistoftodos]{todonotes}
\usepackage{mathtools}
\usepackage{amssymb}

\title{Zeta Function}
\author{Emil Kerimov}
\date{\today}
\begin{document}
\maketitle

\newtheorem{theorem}{Theorem}[section]
\newtheorem{corollary}{Corollary}[theorem]
\newtheorem{lemma}[theorem]{Lemma}
\newtheorem{definition}{Definition}[section]

\section{Definition}

\begin{definition}\label{zeta def}
The Riemann Zeta function is defined as 
$$
\zeta(z) = \sum_{k=1}^{\infty} \frac{1}{k^z}
$$
\end{definition}

We have already shown that \footnote{See Gamma function document}
\begin{equation}
\zeta(2) = \sum_{k=1}^{\infty} \frac{1}{k^2} =\frac{\pi^2}{6}
\end{equation}

\section{Bernoulli Numbers}
Examining the Taylor expansion of $\frac{x}{e^x -1}$, this will be required for the evaluation of the even integer values of $\zeta.$

\begin{definition}\label{bernoulli def}
The Bernoulli numbers $B_n$ are defined as the coefficients of the Taylor expansion of the following function
$$\boxed{
\frac{x}{e^x -1} = \sum_{n=0}^{\infty} \frac{B_n}{n!} x^n
}
$$
\end{definition}

\begin{theorem}
\begin{equation} \label{bernoulli generating}
\sum_{k}^{n} \binom{n+1}{k} B_k = 0  \quad \forall n > 0, \text{ with } B_0 = 1
\end{equation}

Proof
\begin{gather*}
\intertext{Starting from definition \ref{bernoulli def}}
\frac{x}{e^x -1} = \sum_{n=0}^{\infty} \frac{B_n}{n!} x^n
\\
\intertext{Using the Taylor expansion of $e^x$}
\frac{x}{\sum_{k=1}^{\infty} \frac{x^k}{k!}} = \sum_{n=0}^{\infty} \frac{B_n}{n!} x^n
\\
\intertext{Rearranging}
x = \sum_{k=1}^{\infty} \frac{x^k}{k!} \sum_{n=0}^{\infty} \frac{B_n}{n!} x^n
\\
1 = \sum_{k=1}^{\infty} \frac{x^{k-1}}{k!} \sum_{n=0}^{\infty} \frac{B_n}{n!} x
\\
\intertext{Adjusting the indicies}
1 = \sum_{k=0}^{\infty} \frac{x^{k}}{(k+1)!} \sum_{n=0}^{\infty} \frac{B_n}{n!} x^n
\\
\intertext{Using Cauchy's product formula for infinite sums \footnotemark}
1 = \sum_{n=0}^{\infty} 
\underbrace{\sum_{k=0}^{n} 
\underbrace{\Big(\frac{B_k}{k!} x^k \Big)}_{a_k} 
\underbrace{\Big(\frac{x^{n-k}}{(n-k+1)!} \Big)}_{b_{n-k}}}_{c_n}
\\
\intertext{Simplifying}
1 = \sum_{n=0}^{\infty}
\frac{1}{(n+1)!} 
\sum_{k=0}^{n} 
\binom{n+1}{k}
B_k x^k  x^{n-k}
\\
1 = \sum_{n=0}^{\infty}
\frac{x^n}{(n+1)!} 
\sum_{k=0}^{n}  \binom{n+1}{k} B_k 
\intertext{Comparing coefficients of powers of $x$ of both sides we get}
n=0, \quad  1 =  B_0 
\\
n \not =0, \quad 0= \sum_{k=0}^{n}  \binom{n+1}{k} B_k 
\end{gather*}
\end{theorem}

\footnotetext{$\Big( \sum_{n=0}^{\infty} a_n \Big) \Big( \sum_{n=0}^{\infty} b_n \Big) = \sum_{n=0}^{\infty} c_n $ where $c_n = \sum_{k=0}^{n} a_k b_{n-k}$. See proof: TODO ADD}

\subsection{First few Bernoulli numbers}
Using equation \ref{bernoulli generating} with $n=1$ implies $ \binom{2}{0} B_0 + \binom{2}{1} B_1 = 0 = 1 + 2 B_1 $ which implies 
\begin{equation}
B_1 = -\frac{1}{2}
\end{equation}
 
Using equation \ref{bernoulli generating} with $n=2$ implies $ \binom{3}{0} B_0 + \binom{3}{1} B_1 + \binom{3}{2} B_2 = 0 = 1 - \frac{3}{2} + 3 B_2 $ which implies 
\begin{equation}
B_2 = \frac{1}{6}
\end{equation}

Similarly when we apply this equation for increasing values of n
\begin{equation}
B_3 = 0, B_4 = -\frac{1}{30}, B_5 = 0, B_6 = \frac{1}{42}, ...
\end{equation}


\begin{theorem}
In fact we can show that all odd Bernoulli numbers after n=1 are 0
\begin{equation} \label{odd bernoulli} B_n = 0  \quad \forall \text{ odd }n > 1
\end{equation}

Proof
\begin{gather*}
\intertext{Starting from definition \ref{bernoulli def}}
\frac{x}{e^x -1} = \sum_{n=0}^{\infty} \frac{B_n}{n!} x^n
\\
\intertext{Removing the first two terms of the sum}
\frac{x}{e^x -1} = B_0 + B_1 x + \sum_{n=2}^{\infty} \frac{B_n}{n!} x^n\\
\intertext{Rearranging}
\frac{x}{e^x -1}  - 1 + \frac{x}{2}= \sum_{n=2}^{\infty} \frac{B_n}{n!} x^n
\\
\frac{x + xe^x}{2e^x -2}  - 1 = \sum_{n=2}^{\infty} \frac{B_n}{n!} x^n
\\
\intertext{If we can show that the RHS is an even function then we have established the proof}
y(x) = \frac{x + xe^x}{2e^x -2}  - 1
\\
y(-x) = \frac{-x + -xe^{-x}}{2e^{-x} -2}  - 1
\\
\intertext{Simplfying}
y(-x) = \frac{-x -xe^{-x}}{2e^{-x} -2} \cdot \frac{e^x}{e^x}  - 1
\\
y(-x) = \frac{-xe^x -x}{2 -2e^x}  - 1
\\
y(-x) = \frac{x+xe^x}{2e^x - 2}  - 1
\\
y(-x) = y(x)
\\
B_n = 0  \quad \forall \text{ odd }n > 1
\end{gather*}
\end{theorem}

\pagebreak

\section{Even integer values $\zeta(2n)$}

While the odd integer values of $\zeta(n)$ do not have a closed form expression, the even values do.

\begin{theorem}
We can express all even integer values of $\zeta$ using Bernoulli numbers
\begin{equation} \label{even zeta} 
\zeta(2n) = (-1)^{n+1} \frac{(2\pi)^{2n} B_{2n}}{2 \cdot (2n)!}
\end{equation}

Proof
\begin{gather*}
\intertext{Starting from the series representation \footnotemark of cotangent}
\pi x cot(\pi x) = 1 - \sum_{k=1}^{\infty} \zeta(2k) x^{2k}
\\
\intertext{Using the complex exponential representation \footnotemark of cotangent}
i \pi x +  \frac{2i \pi x}{e^{2 i\pi x} -1} = 1 - \sum_{k=1}^{\infty} \zeta(2k) x^{2k}
\\
\intertext{Using the Bernoulli numbers definition \ref{bernoulli def}} 
i \pi x + \sum_{n=0}^{\infty} \frac{B_n}{n!} (2 i\pi x)^n
 = 1 - \sum_{k=1}^{\infty} \zeta(2k) x^{2k}
\\
\intertext{Rearranging and using the first two values of the Bernoulli numbers} 
\\
i \pi x + B_0 + B_1 (2 i \pi x) +  \sum_{n=2}^{\infty} \frac{B_n}{n!} (2 i\pi x)^n
 = 1 - \sum_{k=1}^{\infty} \zeta(2k) x^{2k}
\\
i \pi x + 1 + - i \pi x +  \sum_{n=2}^{\infty} \frac{B_n}{n!} (2 i\pi x)^n
 = 1 - \sum_{k=1}^{\infty} \zeta(2k) x^{2k}
\\
\sum_{n=2}^{\infty} \frac{B_n}{n!} (2 i\pi)^nx^n
 = - \sum_{k=1}^{\infty} \zeta(2k) x^{2k}
\\
\intertext{Using the fact the the odd Bernoulli numbers after $B_1$ are 0} 
\sum_{k=1}^{\infty} \frac{B_{2k}}{(2k)!} (-1)^{k}(2 \pi)^{2k} x^{2k}
 = \sum_{k=1}^{\infty} - \zeta(2k) x^{2k}\ 
\\
\intertext{Comparing coefficients}
\zeta(2k) =  \frac{B_{2k}}{(2k)!} (-1)^{k+1}(2 \pi)^{2k} \quad \forall  k \in \mathbb N_{> 0}
\end{gather*}
\end{theorem}

\addtocounter{footnote}{-1}
\footnotetext{See Euler's Sine Product Formula document}
\stepcounter{footnote}
\footnotetext{$cot(x) = i +  \frac{2i}{e^{2ix} -1}$. See Basic trig functions document}

\subsection{The first few even integer values of $\zeta$}

\begin{gather*}
\zeta(2) = \frac{\pi^2}{6}, 
\zeta(4) = \frac{\pi^4}{90},
\zeta(6) = \frac{\pi^6}{945},
\zeta(8) = \frac{\pi^8}{9450},
\zeta(10) = \frac{\pi^{10}}{93555},
\\
\zeta(12) = \frac{691}{638512875} \pi^{12},
\zeta(14) = \frac{2}{18243225} \pi^{14},
\zeta(16) = \frac{3617}{325641566250} \pi^{16},
\\
\zeta(18) = \frac{43867}{38979295480125} \pi^{18},
\zeta(20) = \frac{174611}{1531329465290625} \pi^{20},
\zeta(22) = \frac{155366}{13447856940643125} \pi^{22},
\\
\zeta(24) = \frac{236364091}{201919571963756521875} \pi^{24},
...
\end{gather*}

\pagebreak

\section{Related functions}

\subsection{Dirichlet Eta Function}

\begin{definition}\label{Dirichlet Eta Function def}
The Dirichlet Eta function is defined as 
$$
\eta(z) = \sum_{k=1}^{\infty} \frac{(-1)^{k+1}}{k^z} = 1 - \frac{1}{2^z} +  \frac{1}{3^z} -  \frac{1}{4^z} + ...
$$
\end{definition}

\begin{theorem}
\begin{equation}
\boxed{
\eta(z) = (1 - 2^{1-s}) \zeta(z)
}
\end{equation}

Proof
\\
\begin{gather*}
\text{Starting from the definition of $\eta(z)$}
\\
\eta(z) 
=
\sum_{k=1}^{\infty} \frac{(-1)^{k+1}}{k^z}
=
\sum_{k=1}^{\infty} \frac{1}{k^z} - 2 \sum_{k=1}^{\infty} \frac{1}{(2k)^z}
\\
\text{Using the definition of $\zeta(z)$}
\\
\eta(z) 
=
\zeta(z) - \frac{2}{2^{z}} \zeta(z)
\\
\eta(z) = (1 - 2^{1-z}) \zeta(z)
\end{gather*}
\end{theorem}


\subsubsection{Values}
\begin{equation}
\eta(1) = \sum_{k=1}^{\infty} \frac{(-1)^{k+1}}{k} = ln(2) \quad \quad \footnotemark 
\end{equation}

\begin{equation} 
\eta(2) = (1 - 2^{1-2}) \cdot \zeta(2) =
\frac{1}{2} \cdot \frac{\pi^2}{6} = \frac{\pi^2}{12} 
\end{equation}

\footnotetext{This can be seen by taking the Taylor-Expansion of $ln(x+1)$ and evaluating at $x=1$}

\pagebreak

\subsection{Dirichlet Lambda Function}

\begin{definition}\label{Dirichlet Lambda Function def}
The Dirichlet Lambda function is defined as 
$$
\lambda(z) = \sum_{k=0}^{\infty} \frac{1}{(2k +1))^z} = 1 + \frac{1}{3^z} +  \frac{1}{5^z} +  \frac{1}{7^z} + ...
$$
\end{definition}

\begin{theorem}
\begin{equation}
\boxed{
\lambda(z) = (1 - 2^{-s}) \zeta(z)
}
\end{equation}

Proof
\\
\begin{gather*}
\text{Starting from the definition of $\lambda(z)$}
\\
\lambda(z) 
=
\sum_{k=0}^{\infty} \frac{1}{(2k +1))^z}
=
\sum_{k=1}^{\infty} \frac{1}{k^z} - \sum_{k=1}^{\infty} \frac{1}{(2k)^z}
\\
\text{Using the definition of $\zeta(z)$}
\\
\lambda(z) 
=
\zeta(z) - \frac{1}{2^{z}} \zeta(z)
\\
\lambda(z) = (1 - 2^{-z}) \zeta(z)
\end{gather*}
\end{theorem}

\subsubsection{Values}

\begin{equation} 
\lambda(2) = (1 - 2^{-2}) \cdot \zeta(2) =
\frac{3}{4} \cdot \frac{\pi^2}{6} = \frac{\pi^2}{8} 
\end{equation}

\pagebreak

\section{Integral Forms}
\subsection{Gamma with Zeta functions}
\begin{theorem}
See the Gamma function document for the definition and properties of the Gamma function.
\begin{equation}
\boxed{
\Gamma(z) \zeta(z) = \int_{0}^{\infty} \frac{u^{z-1}}{e^{u} - 1} du}
\end{equation}

Proof
\\
\begin{gather*}
\text{Using $\Gamma(z) = n^z  \int_{0}^{\infty} e^{-n u} u^{z-1} du$ from the Gamma Function document and rearranging}
\\
\Gamma(z) \frac{1}{n^z} =  \int_{0}^{\infty} e^{-n u} u^{z-1} du
\\
\text{Let n be all integers and take the sum of all equalities}
\\
\sum_{n=1}^{\infty} \Gamma(z) \frac{1}{n^z} 
=
\Gamma(z) \zeta(z)
= 
\sum_{n=1}^{\infty} \int_{0}^{\infty} e^{-n u} u^{z-1} du
\\
\text{Bringing the summation into the integral since this integral converges uniformly}
\\
\Gamma(z) \zeta(z) =  \int_{0}^{\infty} \Big( \sum_{n=1}^{\infty} e^{-n u} \Big) u^{z-1} du
\\
\text{Notice that $\sum_{n=1}^{\infty} e^{-n u}$ is a geometric series if $u>0$}
\\
\Gamma(z) \zeta(z) =  \int_{0}^{\infty} \Big(\frac{e^{-u}}{1-e^{-u}} \Big) u^{z-1} du
\\
\text{Simplfiying by multiplying top and bottom by $e^u$}
\\
\Gamma(z) \zeta(z) =  \int_{0}^{\infty} \frac{u^{z-1}}{e^u -1} du
\end{gather*}
\end{theorem}

\pagebreak

\section{Jacobi Theta Function}

\subsection{Definition}
\begin{definition}\label{Jacobi Theta def}
The Jacobi Theta function is defined as 
$$
\Theta(x) = \sum_{k=-\infty}^{\infty} e^{- \pi n^2 x} 
$$
\end{definition}

\begin{theorem}
The Jacobi Theta function has the following functional form
\begin{equation} \label{Jacobi functional form}
\Theta(x) = \frac{1}{\sqrt{x}} \Theta(\frac{1}{x})
\end{equation}

Proof
\begin{gather*}
\intertext{Using the Poisson Summation Formula \footnotemark}
\Theta(x) = \sum_{n=-\infty}^{\infty} e^{- \pi n^2 x} = \sum_{k=-\infty}^{\infty} 
\int_{-\infty}^{\infty}
e^{- \pi y^2 x} e^{-2 \pi i k y} dy 
=
\sum_{k=-\infty}^{\infty} 
\int_{-\infty}^{\infty}
e^{- \pi y^2 x -2 \pi i k y} dy 
\\
\intertext{Re-writting the exponent}
\\ 
- \pi y^2 x -2 \pi i k y 
= - \pi x (y^2 -2 \pi i \frac{k}{x} y)
= - \pi x (y + i \frac{k}{x})^2 - \pi \frac{k^2}{x}
\\
\intertext{Plugging in the factored exponent, we get}
\\
\Theta(x) 
=
\sum_{k=-\infty}^{\infty} 
\int_{-\infty}^{\infty}
e^{- \pi x (y + i \frac{k}{x})^2 - \pi \frac{k^2}{x}} dy 
=
\sum_{k=-\infty}^{\infty}
e^{- \pi \frac{k^2}{x}} 
\int_{-\infty}^{\infty}
e^{- \pi x (y + i \frac{k}{x})^2} dy
\\
\intertext{After a change of variables $z = y + i \frac{k}{x}$ we can use the Guassian integral identity as proved before, refer to the Guassian integral document, $\int_{-\infty}^{\infty}
e^{- a z^2 } dz = \sqrt{\frac{\pi}{a}} $, we get:}
\\ 
\Theta(x) 
=
\sum_{k=-\infty}^{\infty}
e^{- \pi \frac{k^2}{x}} 
\sqrt{\frac{\pi}{x \pi}}
= \frac{1}{\sqrt{x}} \Theta(\frac{1}{x}) 
\end{gather*}
\end{theorem}

\footnotetext{$\sum_{n=-\infty}^{\infty} f(n)  = \sum_{n=-\infty}^{\infty} \hat{f}(n)$. See proof: TODO ADD}


\section{Functional Form}
\subsection{First Form}

\begin{theorem}
The symmetric functional equation of $\zeta$
\begin{equation} 
\boxed{
\pi^{-\frac{s}{2}} \Gamma(\frac{s}{2}) \zeta(s) = \pi^{- \frac{1-s}{2}} \Gamma(\frac{1-s}{2}) \zeta(1-s)
}
\end{equation}

Proof
\begin{gather*}
TODO 
\end{gather*}
\end{theorem}

\subsection{Second Form}
\begin{theorem}
\begin{equation} 
\boxed{
\zeta(s) = 2^s \pi^{s-1} sin(\frac{\pi s}{2}) \Gamma (1-s) \zeta(1-s)
}
\end{equation}

Proof
\begin{gather*}
TODO 
\end{gather*}
\end{theorem}

\section{Riemann Hypothesis Statement}

\end{document}
