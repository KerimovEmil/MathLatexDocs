\documentclass[a4paper]{article}

\usepackage[english]{babel}
\usepackage[utf8x]{inputenc}
\usepackage{amsmath}
\usepackage{graphicx}
\usepackage[colorinlistoftodos]{todonotes}
\usepackage{mathtools}
\usepackage{amssymb}

\title{Bernoulli Numbers}
\author{Emil Kerimov}
\date{\today}
\begin{document}
\maketitle

\newtheorem{theorem}{Theorem}[section]
\newtheorem{corollary}{Corollary}[theorem]
\newtheorem{lemma}[theorem]{Lemma}
\newtheorem{definition}{Definition}[section]


\section{Bernoulli Numbers}
Examining the Taylor expansion of $\frac{x}{e^x -1}$, this will be required for the evaluation of the even integer values of $\zeta.$

\begin{definition}\label{bernoulli def}
The Bernoulli numbers $B_n$ are defined as the coefficients of the Taylor expansion of the following function
$$\boxed{
\frac{x}{e^x -1} = \sum_{n=0}^{\infty} \frac{B_n}{n!} x^n
}
$$
\end{definition}

\begin{theorem}
\begin{equation} \label{bernoulli generating}
\sum_{k}^{n} \binom{n+1}{k} B_k = 0  \quad \forall n > 0, \text{ with } B_0 = 1
\end{equation}

Proof
\begin{gather*}
\intertext{Starting from definition \ref{bernoulli def}}
\frac{x}{e^x -1} = \sum_{n=0}^{\infty} \frac{B_n}{n!} x^n
\\
\intertext{Using the Taylor expansion of $e^x$}
\frac{x}{\sum_{k=1}^{\infty} \frac{x^k}{k!}} = \sum_{n=0}^{\infty} \frac{B_n}{n!} x^n
\\
\intertext{Rearranging}
x = \sum_{k=1}^{\infty} \frac{x^k}{k!} \sum_{n=0}^{\infty} \frac{B_n}{n!} x^n
\\
1 = \sum_{k=1}^{\infty} \frac{x^{k-1}}{k!} \sum_{n=0}^{\infty} \frac{B_n}{n!} x
\\
\intertext{Adjusting the indicies}
1 = \sum_{k=0}^{\infty} \frac{x^{k}}{(k+1)!} \sum_{n=0}^{\infty} \frac{B_n}{n!} x^n
\\
\intertext{Using Cauchy's product formula for infinite sums \footnotemark}
1 = \sum_{n=0}^{\infty} 
\underbrace{\sum_{k=0}^{n} 
\underbrace{\Big(\frac{B_k}{k!} x^k \Big)}_{a_k} 
\underbrace{\Big(\frac{x^{n-k}}{(n-k+1)!} \Big)}_{b_{n-k}}}_{c_n}
\\
\intertext{Simplifying}
1 = \sum_{n=0}^{\infty}
\frac{1}{(n+1)!} 
\sum_{k=0}^{n} 
\binom{n+1}{k}
B_k x^k  x^{n-k}
\\
1 = \sum_{n=0}^{\infty}
\frac{x^n}{(n+1)!} 
\sum_{k=0}^{n}  \binom{n+1}{k} B_k 
\intertext{Comparing coefficients of powers of $x$ of both sides we get}
n=0, \quad  1 =  B_0 
\\
n \not =0, \quad 0= \sum_{k=0}^{n}  \binom{n+1}{k} B_k 
\end{gather*}
\end{theorem}

\footnotetext{$\Big( \sum_{n=0}^{\infty} a_n \Big) \Big( \sum_{n=0}^{\infty} b_n \Big) = \sum_{n=0}^{\infty} c_n $ where $c_n = \sum_{k=0}^{n} a_k b_{n-k}$. See proof: TODO ADD}

\subsection{First few Bernoulli numbers}
Using equation \ref{bernoulli generating} with $n=1$ implies $ \binom{2}{0} B_0 + \binom{2}{1} B_1 = 0 = 1 + 2 B_1 $ which implies 
\begin{equation}
B_1 = -\frac{1}{2}
\end{equation}
 
Using equation \ref{bernoulli generating} with $n=2$ implies $ \binom{3}{0} B_0 + \binom{3}{1} B_1 + \binom{3}{2} B_2 = 0 = 1 - \frac{3}{2} + 3 B_2 $ which implies 
\begin{equation}
B_2 = \frac{1}{6}
\end{equation}

Similarly when we apply this equation for increasing values of n
\begin{equation}
B_3 = 0, B_4 = -\frac{1}{30}, B_5 = 0, B_6 = \frac{1}{42}, ...
\end{equation}


\begin{theorem}
In fact we can show that all odd Bernoulli numbers after n=1 are 0
\begin{equation} \label{odd bernoulli} B_n = 0  \quad \forall \text{ odd }n > 1
\end{equation}

Proof
\begin{gather*}
\intertext{Starting from definition \ref{bernoulli def}}
\frac{x}{e^x -1} = \sum_{n=0}^{\infty} \frac{B_n}{n!} x^n
\\
\intertext{Removing the first two terms of the sum}
\frac{x}{e^x -1} = B_0 + B_1 x + \sum_{n=2}^{\infty} \frac{B_n}{n!} x^n\\
\intertext{Rearranging}
\frac{x}{e^x -1}  - 1 + \frac{x}{2}= \sum_{n=2}^{\infty} \frac{B_n}{n!} x^n
\\
\frac{x + xe^x}{2e^x -2}  - 1 = \sum_{n=2}^{\infty} \frac{B_n}{n!} x^n
\\
\intertext{If we can show that the RHS is an even function then we have established the proof}
y(x) = \frac{x + xe^x}{2e^x -2}  - 1
\\
y(-x) = \frac{-x + -xe^{-x}}{2e^{-x} -2}  - 1
\\
\intertext{Simplfying}
y(-x) = \frac{-x -xe^{-x}}{2e^{-x} -2} \cdot \frac{e^x}{e^x}  - 1
\\
y(-x) = \frac{-xe^x -x}{2 -2e^x}  - 1
\\
y(-x) = \frac{x+xe^x}{2e^x - 2}  - 1
\\
y(-x) = y(x)
\\
B_n = 0  \quad \forall \text{ odd }n > 1
\end{gather*}
\end{theorem}

\pagebreak

\section{Bernoulli Polynomials}
\begin{definition}\label{bernoulli poly def}
The Bernoulli polynomials $B_n(x)$ are defined as 
$$\boxed{
B_{n}(x) = \sum_{k=0}^{n} {n \choose k} B_k x^{n-k}
}$$
\end{definition}

\subsection{Examples}
\begin{align*}
B_{0}(x) &= B_0 = 1
\\ 
B_{1}(x) &= B_0 x + B_1 = x - \frac{1}{2}
\\ 
B_{2}(x) &= B_0 x^2 + {2 \choose 1} B_1 x + B_2 = x^2 - x +  \frac{1}{6}
\\ 
B_{3}(x) &= B_0 x^3 + {3 \choose 1} B_1 x^2 + {3 \choose 2} B_2 x + B_3 = x^3 - \frac{3}{2} x^2 +  \frac{1}{2} x
\end{align*}

Note that $B_n(x)$ is a polynomial of degree $n$.

\subsection{Useful Properties}
\begin{theorem}
$B_n(0)$
\begin{equation}
\boxed{B_n(0) = B_n}
\end{equation}
Proof
\begin{gather*}
B_n(x) = \sum_{k=0}^{n} {n \choose k} B_k x^{n-k}
\\
B_n(x) = \sum_{k=0}^{n-1} {n \choose k} B_k x^{n-k} + B_n
\\
B_n(0) = \sum_{k=0}^{n-1} {n \choose k} B_k \cdot 0 + B_n = B_n
\end{gather*}
\end{theorem}

\begin{theorem}
$B_n(1)$  % todo format better using piecewise
\begin{equation}
\boxed{B_n(1) = B_n \quad n\neq 1
\quad B_1(1) = -B_1} 
\end{equation}
Proof
\begin{gather*}
B_n(x) = \sum_{k=0}^{n} {n \choose k} B_k x^{n-k}
\\
B_n(1) = \sum_{k=0}^{n} {n \choose k} B_k
\intertext{Using equation \ref{bernoulli generating}, $\sum_{k}^{n} \binom{n+1}{k} B_k = 0$}
\sum_{k}^{n-1} \binom{n}{k} B_k = 0
\intertext{Adding $B_n$ to both sides}
\sum_{k}^{n} \binom{n}{k} B_k = B_n
\end{gather*}
\end{theorem}

\begin{theorem}
Derivative 
\begin{equation}
\boxed{B^{'}_n(x) = n B_{n-1}(x)}
\end{equation}
Proof
\begin{gather*}
B_n(x) = \sum_{k=0}^{n} {n \choose k} B_k x^{n-k}
\intertext{Taking derivatives of both sides}
B^{'}_n(x) = \sum_{k=0}^{n-1} {n \choose k} B_k (n-k) x^{n-k-1}
\\
B^{'}_n(x) = \sum_{k=0}^{n-1} \frac{n! (n-k)}{k! (n-k)!} B_k  x^{n-1-k}
\\
B^{'}_n(x) = \sum_{k=0}^{n-1} \frac{(n-1)! \cdot n}{k! (n-k-1)!} B_k  x^{n-1-k}
\\
B^{'}_n(x) = n \sum_{k=0}^{n-1} {n-1 \choose k} B_k  x^{n-1-k}
\\
B^{'}_n(x) = n B_{n-1}(x)
\end{gather*}
\end{theorem}

\begin{theorem}
Integral
\begin{equation}
\boxed{\int_{x=0}^{1} B_n(x) dx = 0}
\end{equation}
Proof
\begin{gather*}
\intertext{Starting from the derivative equation, using $m=n-1$}
B^{'}_{m+1}(x) = (m+1) \cdot B_{m}(x)
\intertext{Integrating both sides}
\int_{x=0}^{1} B^{'}_{m+1}(x) = B_{m+1}(1) - B_{m+1}(0) = (m+1) \cdot  \int_{x=0}^{1} B_{m}(x)
\intertext{Using the Bernoulli polynomial properties}
B_{m+1}(1) - B_{m+1}(0) = B_{m+1} - B_{m+1} = 0
\intertext{Therefore for $m \neq -1$}
\int_{x=0}^{1} B_{m}(x) = 0
\end{gather*}
\end{theorem}
 
\pagebreak

\section{Euler-Maclaurin formula}

Let $ \{ x \} = x - \lfloor{x}\rfloor $ then

\begin{theorem}
\begin{subequations} \label{Euler-Maclaurin formula}
\begin{align}
        \sum_{k=1}^{n-1} f(k) &= \int_{x=1}^{n} f(x) dx - 
\sum_{k=1}^{m} \frac{B_k}{k!} \cdot \big( f^{(k-1)}(n) - f^{(k-1)}(1) \big) + 
R_{mn} \\
R_{mn} &= \frac{(-1)^{m+1}}{m!} \cdot \int_{x=1}^{n} B_{m}(\{x\}) f^{(m)} (x) dx 
\end{align}
\end{subequations}

Proof
\begin{gather*}
\intertext{See the Darboux Formula paper for proof}
\end{gather*}
\end{theorem}

\subsection{Stirling's Approximation}
\begin{theorem}
%\begin{equation} 
%n! = \Big( \frac{n}{e} \Big)^{n} \sqrt{2 \pi n} \cdot e^{\sum_{k=2}^{m} \frac{B_k}{k \cdot (k-1)} \cdot \frac{1}{n^{k-1}} + O(\frac{1}{n^{m}})} 
%\end{equation}

\begin{equation} 
n! = \Big( \frac{n}{e} \Big)^{n} \sqrt{2 \pi n} \cdot exp\Bigg(\sum_{k=1}^{\lfloor m/2 \rfloor} \frac{B_{2k}}{2k \cdot (2k-1)}  \frac{1}{n^{2k-1}} + O(\frac{1}{n^{m}})\Bigg) 
\end{equation}


Proof
\begin{gather*}
\intertext{See the Gamma function paper for proof}
\end{gather*}
\end{theorem}

\pagebreak

\section{Connection to $\zeta(2n)$}

See the Zeta function document. 

\begin{theorem}
We can express all even integer values of $\zeta$ using Bernoulli numbers
\begin{equation} \label{even zeta} 
\zeta(2n) = (-1)^{n+1} \frac{(2\pi)^{2n} B_{2n}}{2 \cdot (2n)!}
\end{equation}

Proof:
See the Zeta function document. 
\end{theorem}

\subsection{The first few even integer values of $\zeta$}

\begin{gather*}
\zeta(2) = \frac{\pi^2}{6}, 
\zeta(4) = \frac{\pi^4}{90},
\zeta(6) = \frac{\pi^6}{945},
\zeta(8) = \frac{\pi^8}{9450},
\zeta(10) = \frac{\pi^{10}}{93555},
\\
\zeta(12) = \frac{691}{638512875} \pi^{12},
\zeta(14) = \frac{2}{18243225} \pi^{14},
\zeta(16) = \frac{3617}{325641566250} \pi^{16},
\\
\zeta(18) = \frac{43867}{38979295480125} \pi^{18},
\zeta(20) = \frac{174611}{1531329465290625} \pi^{20},
\zeta(22) = \frac{155366}{13447856940643125} \pi^{22},
\\
\zeta(24) = \frac{236364091}{201919571963756521875} \pi^{24},
...
\end{gather*}

\pagebreak


\end{document}
