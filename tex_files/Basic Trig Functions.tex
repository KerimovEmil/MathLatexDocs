\documentclass[a4paper]{article}

\usepackage[english]{babel}
\usepackage[utf8x]{inputenc}
\usepackage{amsmath}
\usepackage{graphicx}
\usepackage{cleveref}
\usepackage{mathtools}
\usepackage{fancyhdr}
\numberwithin{equation}{subsection}
\usepackage[english]{babel}
 
\newtheorem{theorem}{Theorem}[section]
\newtheorem{corollary}{Corollary}[theorem]
\newtheorem{lemma}[theorem]{Lemma}
\newtheorem{definition}{Definition}[section]

\fancyhf{}
\fancyhead[LE,RO]{\thepage}
\fancyhead[CE]{\Author}
\fancyhead[CO]{\Title}
\renewcommand\headrulewidth{0pt}
\pagestyle{fancy}

\author{Emil Kerimov}
%\title{Basic Trig Functions}

\makeatletter
\let\Title\@author


\begin{document}
\section{Basic Inverse Trig Functions}
If we assume we have this as a given.

\begin{definition}
$$e^{ix} = cos(x) + i \cdot sin(x)$$
\end{definition}

%We can prove every other trigonometry identity.

\begin{theorem}
$$sin(x) = \frac{e^{ix} - e^{-ix}}{2i} =-i \frac{e^{ix} - e^{-ix}}{2}$$
Proof
$$e^{-ix} = cos(-x) + i \cdot sin(-x) = cos(x) - i \cdot sin(x)$$
$$e^{ix} - e^{-ix} = 2i \cdot sin(x)$$
$$sin(x) = \frac{e^{ix} - e^{-ix}}{2i} =-i \frac{e^{ix} - e^{-ix}}{2} $$
\end{theorem}

\begin{theorem}
$$cos(x) = \frac{e^{ix} + e^{-ix}}{2}$$
Proof: Same as above. 
\end{theorem}

\begin{theorem}
$$tan(x) = -i \frac{e^{ix} - e^{-ix}}{e^{ix} + e^{-ix}}$$
Proof: Same as above. 
\end{theorem}

\begin{theorem}
$$ cot(x) = i +  \frac{2i}{e^{2ix} -1} $$
Proof
\begin{gather*}
tan(x) = -i \frac{e^{ix} - e^{-ix}}{e^{ix} + e^{-ix}}
\\
cot(x) = i \frac{e^{ix} + e^{-ix}}{e^{ix} - e^{-ix}}
\\
cot(x) = i \frac{(e^{ix} - e^{-ix}) +( 2e^{-ix})}{e^{ix} - e^{-ix}}
\\
cot(x) = i  + 2i \frac{e^{-ix}}{e^{ix} - e^{-ix}}
\\
cot(x) = i  + \frac{2i}{e^{2ix} - 1}
\end{gather*}
\end{theorem}

If we for now ignore the non-uniqueness of the square root and logarithm of complex numbers, we can obtain 
\begin{theorem}
$$arcsin(x) = -i \cdot ln( ix + \sqrt{1- x^2} )$$
Proof: Let $y=arcsin(x)$
$$sin(y) = \frac{e^{iy} - e^{-iy}}{2i}$$
$$sin(arcsin(x)) = x = \frac{e^{iy} - e^{-iy}}{2i}$$
$$ 2i x = e^{iy} - e^{-iy}$$
$$ e^{2iy} - 2ix e^{iy} - 1 = 0$$
$$ e^{iy} = \frac{2ix \pm \sqrt{-4x^2 + 4}}{2}$$
$$ e^{iy} = ix \pm \sqrt{1 -x^2}$$
$$ iy =  ln( ix + \sqrt{1- x^2} )$$
$$ y = \frac{ln( ix + \sqrt{1- x^2})}{i} = -i \cdot ln( ix + \sqrt{1- x^2})$$
$$arcsin(x) = -i \cdot ln( ix + \sqrt{1- x^2} )$$
\end{theorem}

\begin{theorem}
$$arccos(x) = -i \cdot ln(x + \sqrt{ x^2 - 1} )$$
Proof: Same as above.
\end{theorem}

\begin{theorem}
$$arctan(x) = \frac{i}{2} \cdot ln\Big(\frac{i+x}{i-x}\Big)$$
Proof: Same as above.
\end{theorem}

\begin{theorem}
$$arcsec(x) = -i ln \Big( \frac{1}{x}  + \sqrt{1 - \frac{i}{x^2}} \Big)$$
Proof: Same as above.
\end{theorem}

\begin{theorem}
$$arccsc(x) = -i ln \Big( \frac{i}{x}  + \sqrt{1 - \frac{1}{x^2}} \Big)$$
Proof: Same as above.
\end{theorem}

\begin{theorem}
$$arccot(x) = \frac{i}{2} \cdot ln\Big(\frac{x-i}{x+i}\Big)$$
Proof: Same as above.
\end{theorem}

The same can be applied to hyperbolic functions, using

\begin{definition}
$$sinh(x) = \frac{e^x - e^{-x}}{2} $$
\end{definition}

\begin{definition}
$$cosh(x) = \frac{e^x + e^{-x}}{2} $$
\end{definition}

\begin{definition}
$$tanh(x) = \frac{sinh(x)}{cosh(x)} $$
\end{definition}

For each normal trigonometry identities there is a similar hyperbolic trigonometry identity.

The relation of the two are
\begin{definition}
$$cosh(ix) = cos(x)$$
$$sinh(ix) = i \cdot sin(x)$$
$$tanh(ix) = i \cdot tan(x)$$
$$cos(ix) = cosh(x)$$
$$sin(ix) = i \cdot sinh(x)$$
$$tan(ix) = i \cdot tanh(x)$$
\end{definition}

\end{document}