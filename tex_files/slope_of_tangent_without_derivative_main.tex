\documentclass[a4paper]{article}

\usepackage[english]{babel}
\usepackage[utf8x]{inputenc}
\usepackage{amsmath}
\usepackage{graphicx}
\usepackage[colorinlistoftodos]{todonotes}
\usepackage{mathtools}

\title{Finding slope of tangent line without the Derivative}
\author{Emil Kerimov}
\date{\today}
\begin{document}
\maketitle

\section{A Quadratic Function}

Given a quadratic function $ y = ax^2 + bx + c$, to find the slope of the tangent line for all $x$, we need to solve for $m$ such that $y_t = mx+b$ intersects the quadratic function.

\begin{equation}
ax^2 + bx + c = mx+k
\end{equation}

Since we know the quadratic function is strictly convex, we know that the tangent line will only intersect the quadratic curve at one and only one point.

\begin{gather*}
ax^2 + bx + c = mx+k\\
\shortintertext{Rearranging}
ax^2 + (b-m)x + (c-k) = 0\\
\shortintertext{Using the quadratic equation}
x = \frac{m-b \pm \sqrt{(b-m)^2 - 4a(c-k)}}{2a}\\
\shortintertext{Since we know that there is only one intersection between these two curve, we know that the term inside the square root needs to be $0$, hence making both solution of $x$ the same.}
x = \frac{m-b}{2a}\\
\shortintertext{Solving for $m$}
 m = 2ax +b \\
\end{gather*}

Notice that this is the same as the derivative of $ax^2 + bx +c$.

\section{A reciprocal function}
Given a reciprocal function $ y = \frac{a}{x}$, to find the slope of the tangent line for all $x$, we need to solve for $m$ such that $y_t = mx+b$ intersects the quadratic function.

\begin{equation}
\frac{a}{x} = mx+k
\end{equation}

Since we know the reciprocal function is strictly concave, we know that the tangent line will only intersect the reciprocal curve at one and only one point.


\begin{gather*}
\frac{a}{x} = mx+k\\
\shortintertext{Rearranging}
mx^2 + kx - a = 0\\
\shortintertext{Using the quadratic equation}
x = \frac{-k \pm \sqrt{k^2 + 4ma}}{2m}\\
\shortintertext{Since we know that there is only one intersection between these two curve, we know that the term inside the square root needs to be $0$, hence making both solution of $x$ the same.}
x = \frac{-k}{2m}\\
\shortintertext{Substituting for k}
x = \frac{-(\frac{a}{x}-mx)}{2m}\\
\shortintertext{Simplifing}
2mx = mx -\frac{a}{x}\\
\shortintertext{Solving for $m$}
m = -\frac{a}{x^2}\\
\end{gather*}

Notice that this is the same as the derivative of $\frac{a}{x}$.
\end{document}