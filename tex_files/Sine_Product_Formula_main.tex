\documentclass[a4paper]{article}

\usepackage[english]{babel}
\usepackage[utf8x]{inputenc}
\usepackage{amsmath}
\usepackage{graphicx}
\usepackage[colorinlistoftodos]{todonotes}
\usepackage{mathtools}
\usepackage{amssymb}

\title{Euler Sine Product Formula}
\author{Emil Kerimov}
\date{\today}
\begin{document}
\maketitle

\newtheorem{theorem}{Theorem}[section]
\newtheorem{corollary}{Corollary}[theorem]
\newtheorem{lemma}[theorem]{Lemma}
\newtheorem{definition}{Definition}[section]

\section{Sine Product Formula}

\begin{theorem} \label{sine prod formula} 
$\boxed{\frac{sin(x)}{x} = \prod_{k=1}^{\infty} \Big( 1-\frac{x^2}{(k\pi)^2} \Big)}$

Proof
\begin{gather*}
\frac{sin(x)}{x} \text{has roots } x = \pi, -\pi, 2\pi, -2\pi, 3\pi, -3\pi, ... \\
\text{If we treat it as a normal polynomial then it can be written as a product of it's roots like such }\\
\frac{sin(x)}{x} = A \Big(1-\frac{x}{\pi}\Big)\Big(1+\frac{x}{\pi}\Big)
\Big(1-\frac{x}{2\pi}\Big)
\Big(1+\frac{x}{2\pi}\Big)
\Big(1-\frac{x}{3\pi}\Big)
\Big(1+\frac{x}{3\pi}\Big) \cdot \cdot \cdot \\
\text{For justification of why we can do this, see section \ref{rigour}}\\
\frac{sin(x)}{x} = A \Big(1-\frac{x^2}{\pi^2}\Big)
\Big(1-\frac{x^2}{2^2 \pi^2}\Big)
\Big(1-\frac{x^2}{3^2 \pi^2}\Big) \cdot \cdot \cdot \\
\frac{sin(x)}{x} = A \prod_{k=1}^{\infty}
\Big(1-\frac{x^2}{k^2 \pi^2}\Big)
\\
\text{Solving for } A:
\\
\text{From the taylor expansion }
sin(x) = x - \frac{x^3}{3!} + \frac{x^5}{5!} - \frac{x^7}{7!} + ... \\
\lim_{x \to 0} \frac{sin(x)}{x} = \lim_{x \to 0} \frac{x}{x} - \frac{x^2}{3!} + \frac{x^4}{5!} - \frac{x^6}{7!} + ... \\
\lim_{x \to 0} \frac{sin(x)}{x} = 1\\
 A = 1 \\
 \frac{sin(x)}{x} = \prod_{k=1}^{\infty}
\Big(1-\frac{x^2}{k^2 \pi^2}\Big)
\end{gather*}
\end{theorem}

\subsection{Change of variables of the Sine Product Formula}
\begin{theorem} \label{sine prod formula no pi} 
$\boxed{\frac{sin(\pi x)}{\pi x} = \prod_{k=1}^{\infty} \Big( 1-\frac{x^2}{k^2} \Big)}$

Proof\\
Replace $x$ with $x'/\pi$ in Theorem \ref{sine prod formula}
\end{theorem}


\subsection{Basel Problem}
\label{basel}
\subsubsection{Context}
This product form of sine was originally developed by Euler to solve the Basel problem. 
\\
\\
The Basel problem was to determine what value the sum $\sum_{k=1}^{\infty} \frac{1}{k^2}$ converged to. Mathematicians at the time knew that this infinite sum converged due to the various convergence tests, but no one knew the exact value. 
\\
\\
Euler's solution to this open problem was one of the first of many solutions for him to gain a reputation in the international math community.
\\
\\
This problem would later be generalized as a particular value of the zeta function, $\zeta(z) = \sum_{k=1}^{\infty} \frac{1}{k^z}$. Riemann was the first to consider complex value of this function, and even today there is an open millennium prize problem revolving around the zeros of this function. 
\subsubsection{Solution}
\begin{theorem} 
$\boxed{\sum_{k=1}^{\infty} \frac{1}{k^2} =  \frac{\pi^2}{6}}$

Proof\\
\begin{gather*}
\text{Expanding the product form }
\\
\frac{sin(x)}{x} = \prod_{k=1}^{\infty} \Big( 1-\frac{x^2}{(k\pi)^2} \Big) 
\\
\frac{sin(x)}{x} = 1 - \Big( \sum_{k=1}^{\infty} \frac{1}{k^2 \pi^2} \Big) x^2 + O\Big( x^4 \Big)
\\
\text{Expanding using the taylor series }\\
sin(x) = x - \frac{x^3}{3!} + \frac{x^5}{5!} - \frac{x^7}{7!} + ... 
\\
\text{Comparing the coefficient of } x^2 
\\
\sum_{k=1}^{\infty} \frac{1}{k^2 \pi^2} = \frac{1}{3!}
\\
\sum_{k=1}^{\infty} \frac{1}{k^2} = \frac{\pi^2}{6}
\end{gather*}
\end{theorem}


\subsection{Series representation of $cot(\pi x)$}
\begin{theorem} 
$\boxed{ \pi cot(\pi x) = \sum_{k=-\infty}^{\infty} \frac{1}{x + k} 
}$

Proof\\
\begin{gather*}
\text{Starting from Theorem \ref{sine prod formula no pi}}
\\
\frac{sin(\pi x)}{\pi x} = \prod_{k=1}^{\infty} \Big( 1-\frac{x^2}{k^2} \Big)
\\
\intertext{The idea is to take the derivative of the log of both sides. \footnotemark}
\text{First taking the log of both sides}
\\
ln(sin(\pi x)) = ln(\pi x) + \sum_{k=1}^{\infty} ln \Big( 1-\frac{x^2}{k^2} \Big)
\\
\text{Now taking the derivative of both sides w.r.t. $x$ we get}
\\
\pi \frac{cos(\pi x)}{sin(\pi x)} = \frac{1}{x} + \sum_{k=1}^{\infty} \frac{\frac{-2x}{k^2}}{\Big( 1-\frac{x^2}{k^2} \Big)} 
\\
\intertext{Simplifying} 
\pi \cdot cot(\pi x) = \frac{1}{x} 
+ 2x \sum_{k=1}^{\infty} \frac{1}{x^2-k^2}
\\
\text{Note   } \frac{1}{x+k} + \frac{1}{x-k} = \frac{x-k + x+k}{x^2-k^2} = \frac{2x}{x^2-k^2}
\\
\intertext{Plugging this into the series expression} 
\pi \cdot cot(\pi x) = \frac{1}{x} 
+ \sum_{k=1}^{\infty} \frac{1}{x+k} + \frac{1}{x-k}
\\
\intertext{Cleaning up the indicies}
\pi \cdot cot(\pi x) = \sum_{k=-\infty}^{\infty} \frac{1}{x+k}
\end{gather*}
\end{theorem}

\footnotetext{This is similar to the definition of the Di-Gamma function $\Psi$, see Gamma function document.}


\begin{theorem} \label{cot series zeta}
$\boxed{x \pi \cot(\pi x) = 1 - 2 \sum_{k=1}^{\infty} \zeta(2k) x^{2k}
}$

Proof\\
\begin{gather*}
\text{Starting from previous intermediate result}
\\
\pi cot(\pi x) = \frac{1}{x} + \sum_{k=1}^{\infty} \frac{\frac{-2x}{k^2}}{\Big( 1-\frac{x^2}{k^2} \Big)} 
\\
\pi x \cdot cot(\pi x) = 1 + \sum_{k=1}^{\infty} \frac{\frac{-2x^2}{k^2}}{\Big( 1-\frac{x^2}{k^2} \Big)} 
\\
\intertext{Using the infinite geometric series formula $\frac{1}{1-x} = \sum_{n=0}^{\infty} x^n $ for $|x| < 1$}
\pi x \cdot cot(\pi x) = 1 + \sum_{k=1}^{\infty} \Bigg( \sum_{n=0}^{\infty} \Big( \frac{x^2}{k^2} \Big)^n \Bigg) \frac{-2x^2}{k^2}  
\\
\intertext{Simplfying}
\pi x \cdot cot(\pi x) = 1 - 2 \sum_{k=1}^{\infty} \Bigg( \sum_{n=0}^{\infty} \Big( \frac{x^2}{k^2} \Big)^{n+1} \Bigg)
\\
\pi x \cdot cot(\pi x) = 1 - 2 \sum_{k=1}^{\infty} \Bigg( \sum_{n=1}^{\infty} \Big( \frac{x}{k} \Big)^{2n} \Bigg)
\\
\intertext{Interchanging the summations since both series are absolutly convergent \footnotemark}
\pi x \cdot cot(\pi x) = 1 - 2 \sum_{n=1}^{\infty} \Bigg( \sum_{k=1}^{\infty} \Big( \frac{1}{k^{2n}} \Big) \Bigg) x^{2n}
\\
\intertext{Using the definition of $\zeta(x) = \sum_{k=1}^{\infty} \frac{1}{k^x}$}
\pi x \cdot cot(\pi x) = 1 - 2 \sum_{n=1}^{\infty} \zeta(2n) x^{2n}
\end{gather*}
\end{theorem}

\footnotetext{Technically this needs to be shown.}


\subsection{Wallis Product}
\label{wallis product}
\begin{theorem} 
$\boxed{ 
\prod_{n=1}^{\infty} \Big( \frac{2n}{2n-1} \Big) \Big( \frac{2n}{2n+1} \Big) = \frac{\pi}{2}
}$

Proof\\
\begin{gather*}
\text{Starting from Theorem \ref{sine prod formula no pi}}
\\
\frac{sin(\pi x)}{\pi x} = \prod_{k=1}^{\infty} \Big( 1-\frac{x^2}{k^2} \Big)
\\
\intertext{Plugging in $x=\frac{1}{2}$}
\frac{sin(\frac{\pi}{2})}{\frac{\pi}{2}} = 
\frac{2}{\pi} = \prod_{k=1}^{\infty} \Big( 1-\frac{1}{4 k^2} \Big)
\\
\frac{2}{\pi} = \prod_{k=1}^{\infty} \Big( \frac{4 k^2 -1}{4 k^2} \Big)
\\
\frac{2}{\pi} = \prod_{k=1}^{\infty} \Big( \frac{2 k -1}{2 k} \Big) \cdot \Big( \frac{2 k +1}{2 k} \Big)
\\
\frac{\pi}{2} = \prod_{k=1}^{\infty} \Big( \frac{2k}{2 k -1} \Big) \cdot \Big( \frac{2 k }{2 k+1} \Big)
\end{gather*}
\end{theorem}



\section{Rigorous proof of convergence}
\label{rigour}
Euler's original proof did not include proper justification of why we are allowed to write $\sin(x)$ as an infinite product of it's roots. The justification comes from what is known as Weierstrass factorization theorem. 

\subsection{Weierstrass factorization theorem}
Define elementary factors
\begin{equation}
E_{n}(z) = 
\begin{cases}
(1-z) &\text{if } n=0 \\
(1-z) e^{z + z^2/2 + ... + z^n /n} & otherwise 
\end{cases}
\end{equation}

\begin{theorem}
Let f be an entire function, and let $\{a_n\}$ be the non-zero zeros of f repeated according to multiplicity. Suppose also that f has a zero at z=0 of order m $\geq$ 0 (a zero of order m = 0 at z = 0 means $f(0) \not = 0$). Then there exists an entire function g and a sequence of integers  $\{p_n\}$ such that
\begin{equation}
f(z) = z^m e^{g(z)} \prod_{n=1}^{\infty} E_{p_n}(\frac{z}{a_n})
\end{equation}



Proof
TODO
\end{theorem}


\end{document}