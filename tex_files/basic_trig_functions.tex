\documentclass[a4paper]{article}

% Shared Preamble for MathLatexDocs

% --- Core Packages ---
\usepackage[english]{babel}
\usepackage[utf8]{inputenc} % Standard utf8 instead of utf8x
\usepackage{amsmath}
\usepackage{amsthm}
\usepackage{amssymb}
\usepackage{mathtools}
\usepackage{graphicx}
\usepackage{microtype} % Better justification and fewer overfull hboxes

% --- Layout & Design ---
\usepackage[colorinlistoftodos]{todonotes}
\usepackage[colorlinks=true, linkcolor=blue, citecolor=magenta]{hyperref}
\usepackage{cleveref} % Better cross-referencing
\usepackage{tcolorbox} % For styled boxes
\usepackage{tikz} % For drawing diagrams

% --- Fonts ---
% Reverted to default LaTeX font (Computer Modern)
% \usepackage{newpxtext} 
% \usepackage{newpxmath} 

% --- Theorem Environments (Classic Style) ---
\newtheorem{classicthm}{Theorem}[section]
\newtheorem{classicdef}[classicthm]{Definition}
\newtheorem{classiclem}[classicthm]{Lemma}
\newtheorem{theorem}{Theorem}[section]
\newtheorem{corollary}{Corollary}[theorem]
\newtheorem{lemma}[theorem]{Lemma}
\newtheorem{definition}{Definition}[section]

% Wrappers to maintain compatibility with the {Title}{Label} syntax used in files
\newenvironment{mytheorem}[2]{
  \begin{classicthm}[#1]\label{thm:#2}
}{
  \end{classicthm}
}

\newenvironment{mydefinition}[2]{
  \begin{classicdef}[#1]\label{def:#2}
}{
  \end{classicdef}
}

\newenvironment{mylemma}[2]{
  \begin{classiclem}[#1]\label{lem:#2}
}{
  \end{classiclem}
}

% --- Custom Commands ---
\newcommand{\Z}{\mathbb{Z}}
\newcommand{\R}{\mathbb{R}}
\newcommand{\C}{\mathbb{C}}
\newcommand{\N}{\mathbb{N}}


\title{Basic Trig Functions}
\author{Emil Kerimov}
\date{}
\begin{document}
\maketitle

\section{Basic Inverse Trig Functions}
If we assume we have this as a given.

\begin{mydefinition}{Euler's Formula}{eulers-formula}
\begin{equation}
    e^{ix} = \cos(x) + i \cdot \sin(x)
\end{equation}
\end{mydefinition}

%We can prove every other trigonometry identity.

\begin{mytheorem}{Sine Definition}{sine-def}
    \begin{equation}
        \sin(x) = \frac{e^{ix} - e^{-ix}}{2i} = -i \frac{e^{ix} - e^{-ix}}{2}
    \end{equation}

    \begin{proof}
        \begin{gather*}
            e^{-ix} = \cos(-x) + i \cdot \sin(-x) = \cos(x) - i \cdot \sin(x) \\
            e^{ix} - e^{-ix} = 2i \cdot \sin(x) \\
            \sin(x) = \frac{e^{ix} - e^{-ix}}{2i} = -i \frac{e^{ix} - e^{-ix}}{2}
        \end{gather*}
    \end{proof}
\end{mytheorem}

\begin{mytheorem}{Cosine Definition}{cosine-def}
    \begin{equation}
        \cos(x) = \frac{e^{ix} + e^{-ix}}{2}
    \end{equation}
    \begin{proof}
        Same as above.
    \end{proof}
\end{mytheorem}

\begin{mytheorem}{Tangent Definition}{tangent-def}
    \begin{equation}
        \tan(x) = -i \frac{e^{ix} - e^{-ix}}{e^{ix} + e^{-ix}}
    \end{equation}
    \begin{proof}
        Same as above.
    \end{proof}
\end{mytheorem}

\begin{mytheorem}{Cotangent Definition}{cotangent-def}
    \begin{equation}
        \cot(x) = i +  \frac{2i}{e^{2ix} -1}
    \end{equation}
    \begin{proof}
        \begin{gather*}
            \tan(x) = -i \frac{e^{ix} - e^{-ix}}{e^{ix} + e^{-ix}}
            \\
            \cot(x) = i \frac{e^{ix} + e^{-ix}}{e^{ix} - e^{-ix}}
            \\
            \cot(x) = i \frac{(e^{ix} - e^{-ix}) +( 2e^{-ix})}{e^{ix} - e^{-ix}}
            \\
            \cot(x) = i  + 2i \frac{e^{-ix}}{e^{ix} - e^{-ix}}
            \\
            \cot(x) = i  + \frac{2i}{e^{2ix} - 1}
        \end{gather*}
    \end{proof}
\end{mytheorem}

If we for now ignore the non-uniqueness of the square root and logarithm of complex numbers, we can obtain

\begin{mytheorem}{Arcsine Function}{arcsin-def}
    \begin{equation}
        \arcsin(x) = -i \cdot \ln( ix + \sqrt{1- x^2} )
    \end{equation}
    \begin{proof}
        Let $y = \arcsin(x)$
        \begin{gather*}
            \sin(y) = \frac{e^{iy} - e^{-iy}}{2i} \\
            \sin(\arcsin(x)) = x = \frac{e^{iy} - e^{-iy}}{2i} \\
            2i x = e^{iy} - e^{-iy} \\
            e^{2iy} - 2ix e^{iy} - 1 = 0 \\
            e^{iy} = \frac{2ix \pm \sqrt{-4x^2 + 4}}{2} \\
            e^{iy} = ix \pm \sqrt{1 -x^2} \\
            iy =  \ln( ix + \sqrt{1- x^2} ) \\
            y = \frac{\ln( ix + \sqrt{1- x^2})}{i} = -i \cdot \ln( ix + \sqrt{1- x^2}) \\
            \arcsin(x) = -i \cdot \ln( ix + \sqrt{1- x^2} )
        \end{gather*}
    \end{proof}
\end{mytheorem}

\begin{mytheorem}{Arccosine Function}{arccos-def}
    \begin{equation}
        \arccos(x) = -i \cdot \ln(x + \sqrt{ x^2 - 1} )
    \end{equation}
    \begin{proof}
        Same as above.
    \end{proof}
\end{mytheorem}

\begin{mytheorem}{Arctangent Function}{arctan-def}
    \begin{equation}
        \arctan(x) = \frac{i}{2} \cdot \ln\Big(\frac{i+x}{i-x}\Big)
    \end{equation}
    \begin{proof}
        Same as above.
    \end{proof}
\end{mytheorem}

\begin{mytheorem}{Arcsecant Function}{arcsec-def}
    \begin{equation}
        \operatorname{arcsec}(x) = -i \ln \Big( \frac{1 \pm \sqrt{1 - x^2}}{x} \Big)
    \end{equation}
    \begin{proof}
        Same as above.
    \end{proof}
\end{mytheorem}

\begin{mytheorem}{Arccosecant Function}{arccsc-def}
    \begin{equation}
        \operatorname{arccsc}(x) = -i \ln \Big( \frac{i}{x}  + \sqrt{1 - \frac{1}{x^2}} \Big)
    \end{equation}
    \begin{proof}
        Same as above.
    \end{proof}
\end{mytheorem}

\begin{mytheorem}{Arccotangent Function}{arccot-def}
    \begin{equation}
        \operatorname{arccot}(x) = \frac{i}{2} \cdot \ln\Big(\frac{x-i}{x+i}\Big)
    \end{equation}
    \begin{proof}
        Same as above.
    \end{proof}
\end{mytheorem}

The same can be applied to hyperbolic functions, using

\begin{mydefinition}{Hyperbolic Functions}{hyperbolic-def}
\begin{align}
    \sinh(x) & = \frac{e^x - e^{-x}}{2}    \\
    \cosh(x) & = \frac{e^x + e^{-x}}{2}    \\
    \tanh(x) & = \frac{\sinh(x)}{\cosh(x)}
\end{align}
\end{mydefinition}

For each normal trigonometry identities there is a similar hyperbolic trigonometry identity.

The relation of the two are
\begin{mytheorem}{Relation to Circular Functions}{hyperbolic-relation}
    \begin{align*}
        \cosh(ix) & = \cos(x)          & \sinh(ix) & = i \cdot \sin(x)  \\
        \tanh(ix) & = i \cdot \tan(x)  & \cos(ix)  & = \cosh(x)         \\
        \sin(ix)  & = i \cdot \sinh(x) & \tan(ix)  & = i \cdot \tanh(x)
    \end{align*}
\end{mytheorem}

\end{document}
