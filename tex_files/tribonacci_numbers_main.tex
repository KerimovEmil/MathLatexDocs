\documentclass[a4paper]{article}

\usepackage[english]{babel}
\usepackage[utf8x]{inputenc}
\usepackage{amsmath}
\usepackage{graphicx}
\usepackage[colorinlistoftodos]{todonotes}
\usepackage{mathtools}
\usepackage{amssymb}

\title{Tribonacci Numbers}
\author{Emil Kerimov}
\date{\today}
\begin{document}
\maketitle

\newtheorem{theorem}{Theorem}[section]
\newtheorem{corollary}{Corollary}[theorem]
\newtheorem{lemma}[theorem]{Lemma}
\newtheorem{definition}{Definition}[section]

\section{Definition}\label{sec:definition}

The Tribonacci sequence is defined as:
\begin{equation}
\begin{aligned}
t_0 &= 0 \\
t_1 &= 1 \\
t_2 &= 1 \\
t_{n} &= t_{n-1} + t_{n-2} + t_{n-3} \quad \forall \quad n \in \mathbb{Z}^+
\end{aligned}\label{eq:equation2}
\end{equation}

The first few Tribonacci numbers then would be:
\[
0,1,1,2,4,7,13,24,44, 81, \dots
\]

\section{Ratios}\label{sec:basic_ratio}

Taking the ratios of consecutive Tribonacci numbers we find this pattern:

\begin{gather*}
\frac{t_{2}}{t_{1}} = 1 \\
\frac{t_{3}}{t_{2}} = \frac{2}{1}\\
\frac{t_{4}}{t_{3}} = \frac{2}{1}\\
\frac{t_{5}}{t_{4}} = \frac{7}{4}\\
\frac{t_{6}}{t_{5}} = \frac{13}{7}\\
\frac{t_{7}}{t_{6}} = \frac{26}{13}\\
\end{gather*}


\begin{theorem}
 $\lim_{n \to \infty} \frac{t_{n+1}}{t_{n}} = \frac{1}{3} \cdot \left( 1 + \sqrt[3]{19 + 3 \sqrt{33}} +
 \sqrt[3]{19 - 3 \sqrt{33}} \right) $

Proof
TODO
\end{theorem}


\section{Closed Form Expression}\label{sec:closed-form}
Using the characteristic equation of this sequence definition (see Characteristic Equation of recurrence relations),
we have $x^3 -x^2 -x -1 = 0$, with distinct roots of $\alpha, \beta, \gamma.

\begin{theorem} \label{Charac}
\[
t_{n} = c_1 \cdot \alpha^n + c_2 \cdot \beta^n + c_3 \cdot \gamma^n \quad
 \forall  n \in \mathbb{Z}^+
\]
\shortintertext{Where the constants \(c_{1}, c_{2}, c_{3}\) are derived using the initial conditions.}
\end{theorem}

\begin{theorem}
    \[
    t_{n} = \frac{\alpha^n}{-\alpha^2 + 4 \alpha - 1} + \frac{\beta^n}{-\beta^2 + 4 \beta - 1} +
\frac{\gamma^n}{-\gamma^2 + 4 \gamma - 1} \quad
 \forall  n \in \mathbb{Z}^+
    \]
Proof
\begin{gather*}
    \shortintertext{Using the initial conditions we get}\\
    \begin{align*}
     c_1          + c_2          + c_3          = 0  & & (n=0) \\
      \alpha c_1  +  \beta c_2   + \gamma  c_3 = 1 & & (n=1) \\
     \alpha^2 c_1 + \beta^2 c_2  + \gamma^2 c_3 = 1 & & (n=2)  \\
    \end{align*}
    \shortintertext{Re-writing this as a matrix we get a classic vandermonde matrix}
    \begin{bmatrix*}
        1 & 1 & 1\\
        \alpha & \beta & \gamma\\
        \alpha^2 & \beta^2 & \gamma^2\\
    \end{bmatrix*}
    \begin{bmatrix*}
        c_{1}\\
        c_{2}\\
        c_{3}\\
    \end{bmatrix*}
    =
    \begin{bmatrix*}
        0\\
        1\\
        1\\
    \end{bmatrix*}
    \shortintertext{Using Gaussian elimination we can solve for the coefficients}
    \left(\hspace{-5pt}\begin{array}{ccc|c}
  1 & 1 & 1 & 0 \\
  \alpha & \beta & \gamma & 1 \\
  \alpha^2 & \beta^2 & \gamma^2 & 1
\end{array}\hspace{-5pt}\right)
    \sim
    \left(\hspace{-5pt}\begin{array}{ccc|c}
  1 & 1 & 1 & 0 \\
  0 & \beta - \alpha & \gamma - \alpha & 1 \\
  0 & \beta^2 - \alpha^2 & \gamma^2 - \alpha^2 & 1
\end{array}\hspace{-5pt}\right)
    \sim
        \left(\hspace{-5pt}\begin{array}{ccc|c}
  1 & 1 & 1 & 0 \\
  0 & 1 & \frac{\gamma - \alpha}{\beta - \alpha} & \frac{1}{\beta - \alpha} \\
  0 & 1 & \frac{\gamma^2 - \alpha^2}{\beta^2 - \alpha^2} & \frac{1}{\beta^2 - \alpha^2}
  \end{array}\hspace{-5pt}\right)
    \\
                                   \sim
        \left(\hspace{-5pt}\begin{array}{ccc|c}
  1 & 1 & 1 & 0 \\
  0 & 1 & \frac{\gamma - \alpha}{\beta - \alpha} & \frac{1}{\beta - \alpha} \\
  0 & 0 & \frac{\gamma^2 - \alpha^2}{\beta^2 - \alpha^2} -\frac{\gamma - \alpha}{\beta - \alpha}
  & \frac{1}{\beta^2 - \alpha^2} -  \frac{1}{\beta - \alpha}
\end{array}\hspace{-5pt}\right)
    \shortintertext{Therefore we can write \(c_3\):}
    \begin{align*}
    c_3 &= \frac{\frac{1}{\beta^2 - \alpha^2} -  \frac{1}{\beta - \alpha}}{\frac{\gamma^2 - \alpha^2}{\beta^2 - \alpha^2} -\frac{\gamma - \alpha}{\beta - \alpha}} \\
    c_3 &= \frac{(\beta - \alpha) - (\beta^2 - \alpha^2)}{(\gamma^2 - \alpha^2)(\beta - \alpha) - (\gamma - \alpha)(\beta^2 - \alpha^2)} \\
    c_3 &= \frac{\beta - \alpha}{(\beta - \alpha)(\gamma - \alpha)} \cdot \frac{1 - (\beta + \alpha)}{(\gamma + \alpha) - (\beta + \alpha)} \\
    c_3 &= \frac{1 - (\beta + \alpha)}{(\gamma - \beta)(\gamma - \alpha)} \\
    c_3 &= \frac{1 - (\beta + \alpha + \gamma) + \gamma}{(\gamma - \beta)(\gamma - \alpha)} \\
    \end{align*}
\shortintertext{Since \(\alpha, \beta \) and \( \gamma \) satisfy \(x^3 - x^2 - x - 1 = 0\) we get the following identities}
\alpha + \beta + \gamma = 1 \\
\alpha \cdot \beta \cdot \gamma = 1 \\
\shortintertext{Using these identities we can simplify the expression for \(c_3\):}
\begin{split}
c_3 &= \frac{\gamma}{\gamma^2 - (\alpha + \beta) \gamma + \alpha \beta}\\
c_3 &= \frac{\gamma}{\gamma^2 - (\alpha + \beta + \gamma) \gamma + \gamma^2 + \frac{\alpha \beta \gamma}{\gamma}}\\
c_3 &= \frac{\gamma}{2 \gamma^2 - \gamma + \frac{1}{\gamma}}\\
c_3 &= \frac{1}{2 \gamma - 1 + \frac{1}{\gamma^2}}\\
\shortintertext{Re-writing \( \frac{1}{\gamma^2} \) as \( 2\gamma -  \gamma^2\):}
    1 = \gamma^3 - \gamma^2 - \gamma \\
    \frac{1}{\gamma} = \gamma^2 - \gamma - 1 \\
    \frac{1}{\gamma^2} = \gamma - 1 - \frac{1}{\gamma} \\
    \frac{1}{\gamma^2} = \gamma - 1 -  \gamma^2 + \gamma + 1  \\
    \frac{1}{\gamma^2} = 2\gamma -  \gamma^2 \\
    \shortintertext{Therefore:}
    c_3 &= \frac{1}{-\gamma^2 + 4 \gamma - 1 }\\
        \shortintertext{By symmetry we can see that:}
    c_1 &= \frac{1}{-\alpha^2 + 4 \alpha - 1 }\\
    c_2 &= \frac{1}{-\beta^2 + 4 \beta - 1 }\\
\end{split}
    \shortintertext{Hence:}
        t_{n} = \frac{\alpha^n}{-\alpha^2 + 4 \alpha - 1} + \frac{\beta^n}{-\beta^2 + 4 \beta - 1} +
\frac{\gamma^n}{-\gamma^2 + 4 \gamma - 1} \quad
 \forall  n \in \mathbb{Z}^+
\end{gather*}
\end{theorem}

\pagebreak

\section{Negative Indices}\label{sec:negative}
If we ignore the condition of $n>0$ we can can use the functional form definition of Tribonacci numbers to define
negative indices:

\begin{equation}
t_{n-3} = t_n - t_{n-1} - t_{n-2}\label{eq:equation3}
\end{equation}

Therefore the first few will be:

\begin{align*}
t_{-1} &= t_{2} - t_{1} - t_{0} = 1 - 1 - 0 = 0\\
t_{-2} &= t_{1} - t_{0} - t_{-1} = 1 - 0 - 0 = 1\\
t_{-3} &= t_{0} - t_{-1} - t_{-2} = 0 - 0 - 1 = -1\\
t_{-4} &= t_{-1} - t_{-2} - t_{-3} = 0 - 1 - (-1) = 0\\
t_{-5} &= t_{-2} - t_{-3} - t_{-4} = 1 - (-1) - 0 = 2\\
t_{-6} &= t_{-3} - t_{-4} - t_{-5} = -1 - 0 - 2 = -3\\
\end{align*}

\begin{theorem}
\[
t_{-n} = t_{n-1}^2 - t_{n-2} \cdot t_{n}
\]
Proof by induction
\begin{gather*}
\shortintertext{Assume true for k, k+1, and k+2}
t_{-k} = t_{k-1}^2 - t_{k-2} \cdot t_{k}\\
t_{-k-1} = t_{k-2}^2 - t_{k-3} \cdot t_{k-1}\\
t_{-k-2} = t_{k-3}^2 - t_{k-4} \cdot t_{k-2}\\
\shortintertext{Proving for k+3}
TODO
%\begin{split}
%t_{-k-3} &= t_{k-4}^2 - t_{k-5} \cdot t_{k-3} \\
%&= (t_{k-1} - t_{k-2} - t_{k-3})^2 - t_{k-5} \cdot t_{k-3} \\
%&= t_{k-1}^2 + t_{k-2}^2 + t_{k-3}^2 + 2 t_{k-2} t_{k-3} - 2 (t_{k-1} t_{k-3} + t_{k-1} t_{k-2}) + t_{k-1} t_{k-2} t_{k-3}
%- t_{k-5} \cdot t_{k-3} \\
%&= t_{-k} - t_{-k-1} -t_{-k-2} \\
%		&= t_{k-1}^2 - t_{k-2} \cdot t_{k} -t_{k-2}^2 + t_{k-3} \cdot t_{k-1} - t_{k-3}^2 + t_{k-4} \cdot t_{k-2}\\
%&= t_{-k} - t_{-k-1} -t_{-k-2}\\
%\end{split}
\end{gather*}
\end{theorem}


\end{document}